\documentclass[twocolumn]{article}
\usepackage{amsthm,amsmath}
\RequirePackage{natbib}
%\RequirePackage[authoryear]{natbib}% uncomment this for author-year bibliography
%\RequirePackage{hyperref}
\usepackage[utf8]{inputenc} %unicode support
%\usepackage[applemac]{inputenc} %applemac support if unicode package fails
%\usepackage[latin1]{inputenc} %UNIX support if unicode package fails
\usepackage[flushleft]{threeparttable}
\usepackage{upgreek}

\newcommand{\wolf}[1]{{\textcolor{red}{[Wolf: {#1}]}}}
\newcommand{\elad}[1]{{\textcolor{magenta}{[Wolf: {#1}]}}}

\title{Removing both Futile and Unrealistic Energy-Generating Cycles in Flux Balance Analysis}

\begin{document}
\section*{Introduction}
In \cite{Schilling2000-rn, Beard2002-xt}, three basic categories of extreme pathways were defined:
\begin{description}
\item[Type I] -- Primary systemic pathways
\item[Type II] -- Futile cycles
\item[Type III] -- Internal cycles
\end{description}
More than a decade later, an extension of FBA was proposed that can eliminate all infeasible loops from the solution space \cite{Schellenberger2011-bq}. This method is usually referred to as ``loopless'' FBA (ll-FBA). Later, we proved mathematically that this method is both sound and complete \cite{Noor2012-qb}. Thus, ll-FBA effectively eliminates all flux profiles that contain internal cycles (which violate the first law of thermodynamics) from the solution space, without removing any other solutions. Specifically, Type II extreme pathways (e.g. futile cycles, which only convert stored chemical energy to heat) remain viable solutions. If all reactions of a type II cycle are reversible, then one can construct a cycle comprising the exact same reactions in reverse direction. This new cycle would still be a possible solution, and will ``create'' energy. Although this may seem like a subtle difference, these Energy Generating Cycles (EGCs) do not violate the first law of thermodynamics, and therefore are not removed by ll-FBA. Nevertheless, they are not feasible due to the second law, since they increase the Gibbs free energy (i.e. decrease the entropy of the universe).

EGCs are much more problematic than internal cycles, as their existence can increase the maximal yield of the metabolic network. A typical scenario would be an ATP-coupled cycle that effectively creates ATP from ADP and orthophosphate while all other intermediate compounds are mass balanced. This is equivalent to making ATP without any metabolic cost, which could effectively satisfy the ATP requirement of the biomass function and allow more resources to be diverted to biosynthesis. In well curated models such as the genome-scale \emph{E. coli} model \cite{Carrera2014-ys}, EGCs have been eliminated by manually constraining the directionality of many reactions (specifically, ATP coupled reactions). Although this is an effective way of removing EGCs, it has two major disadvantages: (i) it imposes hard constraints on reactions that might be reversible, and (ii) it is labor intensive and thus not scalable.

\subsection*{Example of an Energy Generating Cycle in iJO1366}
One of the well-known examples for an EGC appears in the latest genome-scale reconstruction of \emph{E. coli} metabolism, denoted iJO1366 \cite{Orth2011-qi}. Orth et al. published the model together with a warning that ``hydrogen peroxide producing and consuming reactions carry flux in unrealistic energy generating loops'' and therefore these reactions are constrained by default to zero. 
See table \ref{table:egc_example} and figure \ref{fig:egc_example}.

\subsection*{Thermodynamic Flux Balance Analysis (TFBA)}

Thermodynamic FBA (also known as Thermodynamic-based Metabolic Flux Analysis) was designed to deal with thermodynamically infeasible flux solutions. However, its widespread adoption has been hampered by the requirement for thermodynamic parameters. The set of equations that describe TFBA are:
\begin{eqnarray}
\textbf{TFBA:~~} && \nonumber\\
\mathbf{S} \cdot \mathbf{v} &=& \mathbf{0}  \label{eq:tfba1} \\
\mathbf{0} ~\leq~ \mathbf{v} &\leq & \mathbf{z} \cdot v_{max} \label{eq:tfba2} \\
\mathbf{S}^\top \cdot \mathbf{\Delta_f G'} &<& K - K\mathbf{z} \label{eq:tfba3} \\
\mathbf{\Delta_f G'}& \geq & \mathbf{\Delta_f G'^\circ} + RT \cdot \ln(\mathbf{B_{low}}) \label{eq:tfba4} \\
\mathbf{\Delta_f G'}& \leq & \mathbf{\Delta_f G'^\circ} + RT \cdot \ln(\mathbf{B_{high}}) \label{eq:tfba5}
\end{eqnarray}
where the constants are the stoichiometric matrix of internal reactions $\mathbf{S_{int}}  \in \mathcal{R}^{m \times r}$  and the vector of standard Gibbs energies of formation $\mathbf{\Delta_f G'^\circ}$ (in units of kJ/mol), the gas constant $R$ = 8.31 J/mol/K and temperature $T$ = 300 K. The variables are the flux vector $\mathbf{v} \in \mathcal{R}_{+}^{r}$, the vector of binary reaction indicators $\mathbf{z} \in \{0,1\}^{r}$, and the vector of Gibbs energies of formation $\mathbf{\Delta_f G'}$. Equation \ref{eq:tfba2} makes sure that $z_i$ would be 1 whenever there is a non-zero flux in reaction $i$. Equation \ref{eq:tfba3} represents the second law of thermodynamics. This is made clear by the fact that multiplyfing the stoichiometric matrix by the vector of chemical potentials gives the vector of changes in reaction Gibbs energies for all reactions in the model, i.e. $\mathbf{\Delta_r G'} \equiv \mathbf{S_{int}}^\top \cdot \mathbf{\Delta_f G'}$. Then, for every reaction whose flux is not zero (which forces $z_i = 1$), the constraint would be $\Delta_r G'_i < 0$ -- as dictated by the second law. $K$ is a large scalar (orders of magnitude larger than all $\mathbf{\Delta_f G'^\circ}$ values), and is there to make sure the constraint is easily satisfied for inactive reactions (i.e. such that $\Delta_r G'_i < K$ will always be true). Finally, Equations \ref{eq:tfba4}-\ref{eq:tfba5} represent the bounds on the Gibbs energies of formation, assuming the concentration of each metabolite is between $B_{low,i}$ and $B_{high,i}$.

While the stoichiometric matrix ($\mathbf{S_{int}}$) and general flux constraints are exactly the same as in the standard FBA formulation, $\mathbf{\Delta_f G'^\circ}$ comes as an additional requirement for running TFBA. Unfortunately, we still lack precise measurement for many of the compounds comprising biochemical networks, and computational methods that estimate $\mathbf{\Delta_f G'^\circ}$ \cite{Jankowski2008-hd,Noor2012-mp,Noor2013-an,Jinich2014-nv} are far from perfect and sometimes introduce significant errors. The fact that TFBA also adds one boolean variable for each reaction in the model definitely doesn't help either, since solving the LP becomes much harder, requires a good MILP solver, and takes much longer than standard FBA. Due to the effort involved, and the unclear benefit of the method, TFBA has not gained a wide audience of users so far.

\subsection*{Loopless Flux Balance Analysis (ll-FBA)}
In light of these caveats, it might be easier to understand why ll-FBA was introduced four years \emph{after} TFBA \cite{Schellenberger2011-bq}. Essentially, the loopless algorithm uses exactly the same MILP design as TFBA, while forgoing the actual thermodynamic values. This way, thermodynamically infeasible internal (Type III) cycles are eliminated, while all other pathways are kept \cite{Noor2012-qb}. The set of equations describing ll-FBA are:
\begin{eqnarray}
\textbf{ll-FBA:~~~~~~~~} && \nonumber\\
\mathbf{S} \cdot \mathbf{v} &=& \mathbf{0} \label{eq:llfba1} \\
\mathbf{0} ~\leq~ \mathbf{v} &\leq & \mathbf{z} \cdot v_{max} \label{eq:llfba2} \\
\mathbf{S} ^\top \cdot \mathbf{\Delta_f G'} &\leq & K - (K+1)\mathbf{z} \label{eq:llfba3}
\end{eqnarray}
where all variables and constants are the same as in equations \ref{eq:tfba1}-\ref{eq:tfba4}, and $\mathbf{\Delta_f G'} \in \mathcal{R}^{m}$. This formal representation of ll-FBA might seem different from Schellenberger et al. \cite{Schellenberger2011-bq}, but is actually equivalent. These changes facilitate the comparison to the TFBA algorithm. First, as in TFBA, we make the stoichiometric model irreversible, by duplicating each reversible reaction into two irreversible ones with opposite directionality. This is a common and convenient way to add thermodynamic and loopless constrains and is used in the COBRA toolbox implementation of ll-FBA. In addition, the original formulation contained a variable $\mathbf{G} \in \mathcal{R}^{r}$ which was constrained to be orthogonal to the null-space of $\mathbf{S_{int}}$. According to the fundamental theorem of linear algebra, the image of $A^\top$ is orthogonal to the null-space of $A$. Therefore, we chose to replace $\mathbf{G}$ with the expression $\mathbf{S_{int}} ^\top \cdot \mathbf{\Delta_f G'}$, which is a general formula for a vector from the image of $\mathbf{S_{int}} ^\top$.

\paragraph{A comparison ll-FBA to TFBA} After establishing the equivalence between Equations \ref{eq:llfba1}-\ref{eq:llfba3} and the original formulation of ll-FBA in \cite{Schellenberger2011-bq}, we can move on to compare them to TFBA (i.e. Equations \ref{eq:tfba1}-\ref{eq:tfba4}). One finds two main differences. First, the upper bound on $G_i$ in Equation \ref{eq:llfba3} is a bit tighter when $z_i = 1$ (i.e. -1 instead of 0). This constraint was added in \cite{Schellenberger2011-bq} to avoid the degenerate solution $G_i = 0$ for all $i$, and the margin (-1) was chosen arbitrarily since it does not affect the results. A more significant difference, is that in ll-FBA, there are no bounds on $\mathbf{\Delta_f G'}$ (i.e. as in Equations \ref{eq:tfba4}-\ref{eq:tfba5}). Therefore, ll-FBA is equivalent to TFBA with infinite concentration bounds (and consequently, the values of $\mathbf{\Delta_f G'^\circ}$ have no effect and can be set arbitrarily to 0).

\paragraph{Limited adoption of ll-FBA} Although ll-FBA requires no extra parameters compared to FBA, and its implementation is streamlined as part of the COBRA toolbox, it has yet to become mainstream. A plausible explanation would be that there is an alternative method for eliminating internal cycles in FBA solutions, which does not require an MILP, and can be easily implemented: after applying additions constraints that  define the relevant solution space (e.g. realizing the maximal biomass yield, or keeping all exchange fluxes constant), find a solution with the minimum sum of absolute (or squared) fluxes \cite{Holzhutter2004-qj}. Implementations of this principle with slight variations have been presented under different names such as parsimonious FBA \cite{Lewis2010-rx, Schuetz2012-sv} or CycleFreeFlux \cite{Desouki2015-lh}.

\section*{Results}

\subsection*{A compromise between ll-FBA and TFBA}
So, is there a version of thermodynamic-based FBA, that doesn't require a large set of extra (unknown) parameters, and still has a clear benefit over the standard tools that ignore thermodynamics? Here, we propose such a compromise, by relaxing the majority of second-law constraints (i.e. Equation \ref{eq:tfba3}) and keeping only a few important ones. We will show that this method, which we denote st-FBA (semi-thermodynamic Flux Balance Analysis), is sufficient to eliminate energy generating cycles, while requiring a relatively small set of heuristic assumptions and thermodynamic constants.

First, one must define a set of energy currency metabolites. Although the definition is somewhat heuristic, most biologists would agree that the following are energy equivalents: ATP, pyrophosphate, and a gradient of protons across the membrane. Other specific energy-carrying currency metabolites can be added to the list if desired. Next, we must bound the chemical potential ($\mathbf{\Delta_f G'}$) of these currency metabolites and all their associated degraded forms (see Table \ref{table:potentials}). Note that we chose the chemical potential at 1 mM concentration in an aqueous solution, which is a typical concentration for co-factors in \emph{E. coli} \cite{Bennett2009-rm}. Although it is possible to use the exact measured concentration of each of these metabolites, the effect on the st-FBA results would be at most very minor. 

Finally, we fix the values in the $\mathbf{G_f}$ vector only for these metabolites from the table, while the rest of the values remain free.
\begin{eqnarray}
\textbf{st-FBA:}~~\nonumber\\
\mathbf{S} \cdot \mathbf{v} &=& \mathbf{0} \label{eq:stfba1}\\
\mathbf{0} ~\leq~ \mathbf{v} &\leq & \mathbf{z} \cdot v_{max} \label{eq:stfba2}\\
\mathbf{S}^\top \cdot \mathbf{\Delta_f G'} &\leq & K - K\mathbf{z} \label{eq:stfba3}\\
\forall i~\textsf{in Table \ref{table:potentials}}\nonumber\\
\Delta_f G'_i & \geq & \Delta_f G'^\circ_i + RT \ln(B_{low,i}) \\
\Delta_f G'_i & \leq & \Delta_f G'^\circ_i + RT \ln(B_{high,i})
\end{eqnarray}
So st-FBA is very similar to TFBA, except that only the energy currency metabolites have predefined bounds on their formation energies. In fact, since the concentrations of these metabolites tend to be tightly controlled by homeostasis, it is recommended to set them to fixed concentrations (i.e. by setting the lower and upper bounds to the same value).

\subsection*{Implementation of st-FBA}
The semi-thermodynamic Flux Balance Analysis algorithm was implemented using COBRApy \cite{Ebrahim2013-vw}.

%%%%%%%%%%%%%%%%%%%%%%%%%%%%%%%%%%%%%%%%%%%%%%
%%                                          %%
%% Backmatter begins here                   %%
%%                                          %%
%%%%%%%%%%%%%%%%%%%%%%%%%%%%%%%%%%%%%%%%%%%%%%

\bibliographystyle{bmc-mathphys} % Style BST file (bmc-mathphys, vancouver, spbasic).
\bibliography{stfba_lib}      % Bibliography file (usually '*.bib' )

\section*{Figures}
  \begin{figure}[h!]
  \label{fig:egc_example}
  \caption{An Energy Generating Cycle.
      If the reaction SPODM is not removed from the model iJO1366, it can
      be used in this unrealistic pathway that generates ATP without any
      external input.}
      \end{figure}

\section*{Tables}

\begin{table}[h!]
\caption{Example of an Energy Generating Cycle that occurs in iJO1366}
\begin{tabular}{l|l}
\label{table:egc_example}
Reaction & Formula\\\hline
MDH & mal\_\_L-c + nad\_c = h\_c + nadh\_c + oaa\_c\\
MOX	& h2o2\_c + oaa\_c = mal\_\_L\_c + o2\_c\\
Htex	& h\_e = h\_p\\
EX\_h\_e & = h\_e\\
SPODM & 2.0 h\_c + 2.0 o2s\_c = h2o2\_c + o2\_c\\
NADH17pp & 4.0 h\_c + mqn8\_c + nadh\_c = mql8\_c + nad\_c + 3.0 h\_p\\
QMO3 & mql8\_c + 2.0 o2\_c = 2.0 h\_c + mqn8\_c + 2.0 o2s\_c\\
ATPS4rpp & adp\_c + pi\_c + 4.0 h\_p = atp\_c + 3.0 h\_c + h2o\_c\\\hline
Total & adp\_c + pi\_c = atp\_c + h2o\_c
\end{tabular}
\end{table}


\begin{table}[h!]
\caption{Table of currency metabolites, their standard Gibbs energies of formation, and bounded by their typical concentrations \cite{Bennett2009-rm}.}
\begin{tabular}{l|c|c|r}
\label{table:potentials}
\textbf{metabolite} & $\mathbf{B_{low}}$ & $\mathbf{B_{high}}$ & $\Delta_f G'^\circ$ [kJ/mol] \\ \hline
ATP & $9.63$ mM & $9.63$ mM & $-2296$ \\
ADP & $0.56$ mM & $0.56$ mM & $-1424$ \\
AMP & $0.28$ mM & $0.28$ mM & $-549$ \\
orthophosphate & $1$ mM & $10$ mM & $-1056$ \\
pyrophosphate & $1$ $\upmu$M & $1$ mM & $-1939$ \\
H$^+$ (cytoplasm) & $10^{-7.6}$ M $^\dagger$ & $10^{-7.6}$ M & $0$ \\
H$^+$ (extracellular) & $10^{-7.0}$ M & $10^{-7.0}$ M & $0$ \\
H$_2$O & 1 & 1 & $-157.6$
\end{tabular}
\begin{tablenotes}
\tiny
\item $\dagger$ Corresponding to pH 7.6, as measured by \citep{Wilks2007-lh}
\end{tablenotes}
\end{table}


\end{document}